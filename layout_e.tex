% %
% LAYOUT_E.TEX - Short description of REFMAN.CLS
%                                       99-03-20
%
%  Updated for REFMAN.CLS (LaTeX2e)
%

% Preamble
\documentclass[twoside,a4paper]{refart}
% importing packages
\usepackage{makeidx}
\usepackage{ifthen}
\usepackage{url}
\usepackage{csquotes}
\usepackage{graphicx}
\usepackage{float}
\usepackage{caption}
\usepackage{hyperref}
\usepackage{titlesec}
\usepackage{todonotes}
\usepackage{subcaption}

% Setting up document
\def\bs{\char'134 } % backslash in \tt font.
\newcommand{\ie}{i.\,e.,}
\newcommand{\eg}{e.\,g..}
\DeclareRobustCommand\cs[1]{\texttt{\char`\\#1}}
\captionsetup{hypcap=false}
% Minipage spacing
\def\bmp{\begin{minipage}{0.48\linewidth}\small} 
\def\emp{\end{minipage}\smallskip}


\title{\LaTeX\ Guide Guide}
\author{
John Pan (jpthek9@gmail.com)\\
\url{https://github.com/SnpM}
}

\date{\url{https://github.com/SnpM/latex-guide-guide.git}\\\today}
\emergencystretch1em  %

\pagestyle{myfootings}
\markboth{\LaTeX\ Guide Guide}%
         {\LaTeX\ Guide Guide}

\makeindex 

\setcounter{tocdepth}{2}

\begin{document}

\maketitle

\begin{abstract}
         \LaTeX\ is a high-quality typesetting standard for academic and professional documents. The system may seem complicated because of its coding syntax, but powerful \LaTeX\ editors such as Overleaf streamline the learning the process. A wealth of high-quality \LaTeX\ templates exist online, enabling authors to focus on content rather than appearance. This guide teaches new \LaTeX\ users how to author an instructional \LaTeX\ document in Overleaf based on the \texttt{refman} template.
\end{abstract}

The \LaTeX\ Guide Guide is part of my final project for the Technical Communication course at Thomas Edison State University, mentored by Dr. Douglas Hoehn. 

\tableofcontents

\newpage


%%%%%%%%%%%%%%%%%%%%%%%%%%%%%%%%%%%%%%%%%%%%%%%%%%%%%%%%%%%%%%%%%%%%
% \newpage after every \section
\newcommand{\sectionbreak}{\clearpage}

\section{Introduction}

\subsection{What is \LaTeX?}
LaTeX (pronounced \enquote{Lah-Tech}) is a document preparation system for technical and scientific documents. LaTeX is not a word processor; instead, it encourages authors to focus on content rather than the appearance of their documents\footnote{\url{https://www.latex-project.org/about/}}. 

\begin{minipage}{\linewidth}
\fbox{\includegraphics[width=\linewidth]{graphics/IntroSource.png}}
\captionof{figure}{\LaTeX\ document example}
\end{minipage}

\LaTeX is an extension of \TeX, TODO: More stuff about LaTeX. Maybe companies using it and practical cases.

LaTeX is suitable for producing high-quality technical documents in academic and professional contexts.

\subsection{Authoring with \LaTeX}
Authoring projects with \LaTeX\ is streamlined with the many tools and templates available online.
\par
Overleaf is a beautiful web-based \LaTeX editor designed for collaboration; it boasts 3.9 million plus users from over 3,600 different institutions\footnote{\url{https://www.overleaf.com/about}}.
The software has many powerful features including Git integration, change tracking, and rich text editing\footnote{\url{https://www.overleaf.com/blog/overleaf-v2-launch-announcement}}.

\begin{minipage}{\linewidth}
\fbox{\includegraphics[width=\linewidth]{graphics/OverleafProduct.png}}
\captionof{figure}{Feature demonstration of Overleaf}
%https://www.overleaf.com/for/press/resources
\end{minipage}

\texttt{refman} is a collection of document classes used for writing technical reference manuals\footnote{\url{https://ctan.org/pkg/refman?lang=en}}; it includes formatted structure for document title page, table of contents, sections, and more. As a template, \texttt{refman} provides a polished starting point for technical reference manuals and guides.

\begin{minipage}{\linewidth}
\fbox{\includegraphics[width=\linewidth]{graphics/refman.PNG}}
\captionof{figure}{Title page of compiled \texttt{refman}}
%http://ctan.math.illinois.edu/macros/latex/contrib/refman/layout_e.pdf
\end{minipage}


By using Overleaf and \texttt{refman}, authors can quickly create formatted instructional guides with the power of \LaTeX.

\subsection{Using this Guide}
This document serves as a detailed guide for using \LaTeX, Overleaf, and \texttt{refman} to author an instructional guide. The guide sections are designed to be followed progressively to address the following topics: project setup, basic content-writing, extended design, and resources for learning more. New \LaTeX\ users will acquire a solid foundation of workflow and syntax to begin creating practical \LaTeX\ documents.
\par
TODO: Stuff about using this guide

\section{Getting Started}
The Getting Started section explains how to create a new project based on the \texttt{refman} template, then use Overleaf's interface to configure the project. Before starting, ensure you have an Overleaf account by registering at \url{https://www.overleaf.com/register}.

\begin{minipage}{\linewidth}
\fbox{\includegraphics[width=\linewidth]{graphics/RegisterOverleaf.PNG}}
\captionof{figure}{Overleaf registration site \texttt{refman}}
\end{minipage}

\subsection{Requirements Setup}


\subsection{Creating Project}

\subsection{Main Screen}

\subsection{Configuring Project}



\section{Basic Writing}
This section introduces basic \LaTeX\ syntax. Learn how to configure the project Preamble then write your first section in \LaTeX\ code!
\subsection{Basic LaTeX}
Authoring with \LaTeX\ is similar to writing a program: type code to define program behavior, compile program into consumable file, then open file and enjoy! There are four key principles to consider when authoring documents with \LaTeX:
\begin{enumerate}
    \item Content is written in plaintext.
    \item Formatting and features are applied with commands.
    \item Packages provide extra text, graphics, and presentation commands.
    \item \LaTeX code is typically compiled into a read-only professional-looking PDF.
\end{enumerate}
With these principles in mind, let's start learning some basic \LaTeX\ syntax and commands\footnote{Based on existing instructional material: http://www.rpi.edu/dept/arc/training/latex/class-slides-pc.pdf}.
%TODO: Cite this
\par
The \verb|"\"| backslash character is used to begin all \LaTeX\ commands. E.g.
\verb|\LaTeX| compiles to \LaTeX.
\par
Some commands take input enclosed in curly braces. E.g.
\verb|\textit{}{Some italicized text}| compiles to \textit{Some italicized text}.
\par
{\Large\textbf{Common commands include:}\\}
\verb|\\|, \verb|\newline|, \verb|\par|\\
Various ways to generate new lines.
\par
\verb|\newpage|\\
Insert page break to start text on new page.
\par
\verb|\textit{text}|, \verb|\textbf{text}|, \verb|\texttt{text}|\\
Modifies text to be \textit{italicized}, \textbf{bold}, or \texttt{scientific}.
\par
\verb|\usepackage{<packageName>}|\\
Define a \LaTeX\ package to use for modular commands and behaviors in your project.
\par
\verb|\chapter{<title>}|, \verb|\section{<title>}|, \verb|\subsection{<title>}|\\
Mark the beginning of new chapters/sections/subsections.
\par
\verb|\begin{document}| and \verb|\end{document}|\\
Define area for all the text of the document.
\par
\verb|\begin{abstract}| and \verb|\end{abstract}|\\
Define area for the abstract of the document.
\par
\verb|\title{<document title>}|, \verb|\author{<author>}|, \verb|\date{<date>}|\\
Define the document's title, author, and date.
\par
\verb|\maketitle|\\
Print the document's title, author, and date.
\par
\verb|\markboth{left title}{right title}|\\
Defines the headings or footings on either side of the page.
\par
{\Large\textbf{Certain characters have special meanings in \LaTeX.}}
\par
\begin{tabular}{r|l}
\textbf{Char} & \textbf{Meaning} \\
\verb|%|& Parameter in a macro; also used in tables \\ 
\verb|$|& Used to begin and end math mode \\
\verb|%|& Used for comments in the source file \\
\verb|&|& Tab mark, used in alignments and tables \\
\end{tabular}
\par
See comprehensive guide about \LaTeX\ commands and characters at \url{http://www.rpi.edu/dept/arc/docs/latex/latex-intro.pdf}. In the next subsection, you will use basic knowledge of \LaTeX\ to configure the Preamble of your project.

\subsection{The Preamble}
We previously set the \texttt{Main Document} of the project in Overleaf to
\path{layout_e.tex}. Locate this file and open it to begin adding content to your project.

\begin{minipage}{\linewidth}
\centering
\fbox{\includegraphics[width=\linewidth]{graphics/MainDocumentPreamble.PNG}}
\captionof{figure}{Locating \detokenize{layout_e.tex file}}
\label{fig:Preamble}
\end{minipage}

Notice the highlighted section in figure \ref{fig:Preamble}. The area of from the start of the document to the \verb|\begin{document}| command is the \texttt{Preamble}: the first part a file where you describe \LaTeX\ document type, commands, and more information about the document.

We will configure the Title, Author, Date, Packages Used, and Formatting of the \LaTeX\ document in he Preamble using \LaTeX\ syntax and commands.

\subsubsection{}
To begin configuring the \texttt{preamble}, extend the available features in your \LaTeX\ project by adding the following commands after \verb|\usepackage{ifthen}|.
\begin{verbatim}
\usepackage{url}
\usepackage{csquotes}
\usepackage{graphicx}
\usepackage{float}
\usepackage{caption}
\usepackage{hyperref}
\usepackage{titlesec}
\usepackage{todonotes}
\usepackage{subcaption}
\usepackage{xcolor}
\usepackage{nameref}
\usepackage{listings}
\usepackage{verbatim}
\usepackage[T1]{fontenc}
\end{verbatim}

\subsection{Modifying Section}

\section{Extended Writing}
More advanced features
\subsection{Modularize Project}
\subsection{Add Graphics}
\subsection{Add References}
\subsection{Advanced Formatting}

\section {Resources}
Stuff about online resources.
\subsection{Templates}
\subsection{Further Learning}

%%%%%%%%%%%%%%%%%%%%%%%%%%%%%%%%%%%%%%%%%%%%%%%%%%%%%%%%%%%%%%%%%%%%%%

\input lay_e2

%%%%%%%%%%%%%%%%%%%%%%%%%%%%%%%%%%%%%%%%%%%%%%%%%%%%%%%%%%%%%%%%%%%%%%

\printindex

\end{document}
