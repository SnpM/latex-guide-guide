% %
% LAYOUT_E.TEX - Short description of REFMAN.CLS
%                                       99-03-20
%
%  Updated for REFMAN.CLS (LaTeX2e)
%

% Preamble
\documentclass[twoside,a4paper]{refart}
% importing packages
\usepackage{makeidx}
\usepackage{ifthen}
\usepackage{url}
\usepackage{csquotes}
\usepackage{graphicx}
\usepackage{float}
\usepackage{caption}
\usepackage{hyperref}
\usepackage{titlesec}
\usepackage{todonotes}

% Setting up document
\def\bs{\char'134 } % backslash in \tt font.
\newcommand{\ie}{i.\,e.,}
\newcommand{\eg}{e.\,g..}
\DeclareRobustCommand\cs[1]{\texttt{\char`\\#1}}
\captionsetup{hypcap=false}
% Minipage spacing
\def\bmp{\begin{minipage}{0.48\linewidth}\small} 
\def\emp{\end{minipage}\smallskip}


\title{\LaTeX\ Guide Guide}
\author{
John Pan (jpthek9@gmail.com)\\
\url{https://github.com/SnpM}
}

\date{\url{https://github.com/SnpM/latex-guide-guide.git}\\\today}
\emergencystretch1em  %

\pagestyle{myfootings}
\markboth{\LaTeX\ Guide Guide}%
         {\LaTeX\ Guide Guide}

\makeindex 

\setcounter{tocdepth}{2}

\begin{document}

\maketitle

\begin{abstract}
         \LaTeX\ is a high-quality typesetting standard for academic and professional documents. The system may seem complicated because of its coding syntax, but powerful \LaTeX\ editors such as Overleaf streamline the learning the process. A wealth of high-quality \LaTeX\ templates exist online, enabling authors to focus on content rather than appearance. This guide teaches new \LaTeX\ users how to author an instructional \LaTeX\ document in Overleaf based on the \texttt{refman} template.
\end{abstract}

The \LaTeX\ Guide Guide is part of my final project for the Technical Communication course at Thomas Edison State University, mentored by Dr. Douglas Hoehn. 

\tableofcontents

\newpage


%%%%%%%%%%%%%%%%%%%%%%%%%%%%%%%%%%%%%%%%%%%%%%%%%%%%%%%%%%%%%%%%%%%%
% \newpage after every \section
\newcommand{\sectionbreak}{\clearpage}
\section{Introduction}

%Cut Subsections%

\subsection{What is \LaTeX?}
LaTeX (pronounced \enquote{Lah-Tech}) is a document preparation system for technical and scientific documents. LaTeX is not a word processor; instead, it encourages authors to focus on content rather than the appearance of their documents\footnote{\url{https://www.latex-project.org/about/}}. 

\begin{minipage}{\linewidth}
\fbox{\includegraphics[width=\linewidth]{graphics/IntroSource.png}}
\captionof{figure}{\LaTeX\ document example}
\end{minipage}

\LaTeX is an extension of \TeX, TODO: More stuff about LaTeX. Maybe companies using it and practical cases.

LaTeX is suitable for producing high-quality technical documents in academic and professional contexts.

\subsection{Authoring with \LaTeX}
Authoring projects with \LaTeX\ is streamlined with the many tools and templates available online.
\par
Overleaf is a beautiful web-based \LaTeX editor designed for collaboration; it boasts 3.9 million plus users from over 3,600 different institutions\footnote{\url{https://www.overleaf.com/about}}.
The software has many powerful features including Git integration, change tracking, and rich text editing\footnote{\url{https://www.overleaf.com/blog/overleaf-v2-launch-announcement}}.

\begin{minipage}{\linewidth}
\fbox{\includegraphics[width=\linewidth]{graphics/OverleafProduct.png}}
\captionof{figure}{Feature demonstration of Overleaf}
%https://www.overleaf.com/for/press/resources
\end{minipage}

\texttt{refman} is a collection of document classes used for writing technical reference manuals\footnote{\url{https://ctan.org/pkg/refman?lang=en}}; it includes formatted structure for document title page, table of contents, sections, and more. As a template, \texttt{refman} provides a polished starting point for technical reference manuals and guides.

\begin{minipage}{\linewidth}
\fbox{\includegraphics[width=\linewidth]{graphics/refman.PNG}}
\captionof{figure}{Title page of compiled \texttt{refman}}
%http://ctan.math.illinois.edu/macros/latex/contrib/refman/layout_e.pdf
\end{minipage}


By using Overleaf and \texttt{refman}, authors can quickly create formatted instructional guides with the power of \LaTeX.

\subsection{Using this Guide}
This document serves as a detailed guide for using \LaTeX, Overleaf, and \texttt{refman} to author an instructional guide. The guide sections are designed to be followed progressively to address the following topics: project setup, basic content-writing, extended design, and resources for learning more. New \LaTeX\ users will acquire a solid foundation of workflow and syntax to begin creating practical \LaTeX\ documents.
\par
TODO: Stuff about using this guide


%%%%%%%%%% %%%%%%%%%%%%%% %%%%%%%%%%%% %%%%%%%%%%%% %%%%%%%% %%%%%%%%%%%%

\subsection{A more complex example (Reference Manual)}
\label{refmanex}

We want to create a layout similar to the one used in the \textit{PostScript 
Reference Manual}, with a wide left margin for headings and margin 
notes and a small margin at the right and bottom.

\subsubsection{Page layout}

To define the new layout we use the commands described in the \LaTeX\ 
manual. For full details see the file \texttt{refman.dtx}.

%\pagebreak[2]

\marginlabel{Horizontal:}
First we define two new names for length that we will use often:

\cs{fullwidth} is the width of the whole page minus a margin of 1~inch 
on every side. 
%
\begin{displaymath}
\texttt{fullwidth} = \texttt{paperwidth} - 2\ \textrm{inch}
\end{displaymath}
%
From this the width of the text is calculated.
%
\begin{displaymath}
\texttt{textwidth} = \texttt{fullwidth} \times \texttt{textfraction}
\end{displaymath}
%
\cs{leftmarginwidth} is the width of the left margin that will be 
used for headings and margin notes. 
\begin{displaymath}
\texttt{leftmarginwidth} = \texttt{fullwidth} - \texttt{textwidth}
\end{displaymath}
%
This is a little more difficult in reality because the lengths have to 
be rounded to full points and a possible two column layout -- as used 
in the index -- must be taken into consideration.

\marginlabel{Vertical:}
The vertical layout is a little more difficult because you have to 
deal with the page header and footer. 
%
\begin{displaymath}
\texttt{textheight} = \texttt{paperheight} - 2.5\ \textrm{inch}
\end{displaymath}
%
The result of this calculation is rounded to full lines. Depending on 
the page style -- headings or footings -- it is shifted up or down by one 
line.

\vspace{0pt plus 1cm}

\subsubsection{Section headings}

The headings have to be modified to make them extend into the left 
margin.

In file \texttt{classes.dtx} we find the \verb|\@startsection| command 
that defines the layout of the headings. Only parameters 4 to 6 are 
relevant for us: parameter 4 is the space above and parameter 5 the 
space below the section. The 6th parameter does the actual formatting.

This is the original definition:\nopagebreak
\begin{verbatim}
\newcommand\section{\@startsection
             {section}{1}{\z@}%
             {-3.5ex plus -1ex minus -.2ex}%
             {2.3ex plus .2ex}%
             {\normalfont\Large\bfseries}}
\end{verbatim}

The commands for sub- and subsubsections are similar. Note that the 
measures are all in \texttt{ex}, thus depending on the font size used.

We define a new command \cs{secshape} to format the headings. This 
command uses the whole width of the page for the heading. To discourage 
hyphenation of the heading we give it a high penalty. This still 
allows hyphenation when absolutely necessary.

\begin{verbatim}
\newcommand\secshape{%
               \leftskip=-\leftmarginwidth%
               \rightskip=\@flushglue%
               \hyphenpenalty=2000}
\end{verbatim}

This command is inserted into the 6th parameter of \verb|\@startsection|.

Since the headings now extend into the left margin, we can use a 
smaller font and reduce the space between the text and the heading. The 
new definition looks like the following:\nopagebreak
\begin{verbatim}
\newcommand\section{\@startsection
             {section}{1}{\z@}%
             {-2ex plus -1ex minus -.2ex}%
             {0.5ex plus .2ex}%
             {\secshape\normalfont\large\bfseries}}
\end{verbatim}

\vspace{0pt plus 1cm}

\subsubsection{Setting the margin notes}

The margin notes should always appear on the left side of the text. 
The normal layout puts them into the outer margin in twoside layout.

The file \texttt{latex.dtx} contains the definition of the 
\verb|\@addmarginpar| command which is responsible for the margin 
notes. We don't have to understand the whole definition; the important 
part is the internal variable \verb|\@tempcnta| that is either 
\verb|\@ne| (1) when the note should appear on the right side of the 
text or \verb|\m@ne| ($-1$) when it should appear on the left side.

This is done by the following lines:
\begin{verbatim}
    \@tempcnta\@ne
    \if@twocolumn
        \if@firstcolumn \@tempcnta\m@ne \fi
    \else
      \if@mparswitch
         \ifodd\c@page \else\@tempcnta\m@ne \fi
      \fi
      \if@reversemargin \@tempcnta -\@tempcnta \fi
    \fi
\end{verbatim}

which we simply replace by:
\begin{verbatim}
    \@tempcnta\m@ne
\end{verbatim}

The remaining lines that handle the setting of the margin note 
depending on the parameter \verb|\@tempcnta| are left unchanged.

\subsubsection{Extensions}

The definitions described above are sufficient for simple applications 
but in practical use one may want some additional commands.  
\seealso{Appendix \ref{refman}} You will find the description for the whole new 
class in the appendix~\ref{refman}.

%%%%%%%%%%%%%%%%%%%%%%%%%%%%%%%%%%%%%%%%%%%%%%%%%%%%%%%%%%%%%%%%%%%%%%

\input lay_e2

%%%%%%%%%%%%%%%%%%%%%%%%%%%%%%%%%%%%%%%%%%%%%%%%%%%%%%%%%%%%%%%%%%%%%%

\printindex

\end{document}
